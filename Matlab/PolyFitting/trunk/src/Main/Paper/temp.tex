\documentclass[12pt,letterpaper]{article}

\usepackage{amsmath}
\usepackage{mathrsfs}
\usepackage[pdftex]{color,graphicx}
\usepackage{verbatim}
\usepackage{setspace}
\usepackage{subfig}
\usepackage{algorithm}
\usepackage{algorithmicx}
\usepackage{algpseudocode}

\begin{document}

%%%%%%%%%%%%%%%%%%%%%%%%%%%%%%%%
%%%%%%   TESTING  %%%%%%%%%%%%%%
%%%%%%%%%%%%%%%%%%%%%%%%%%%%%%%%
\section{Error Measurement}

The idea for doing error measurement is to measure the 2D error on each unique sliced
image. Some key things need to be addressed:\newline


1. The interface between C project and SketchUp rendering. \newline

The C code should be able to output a file containing error number for each small piece of faces.
The height of a face is the interval of slices (1/1000). The wide of a face is user-defined.
Say ($w = 10$ pixels).

The SU can first load the whole model. Then, for each height, reload the error faces on top of the
SU model. The interface of this file could be \newline
$Layer \; \#$ \newline
$P1 \; P2 \; =>$ \newline
$P1 \; PP_0 \; Value$ \newline
$PP_0 \; PP_1 \; Value$ \newline 
$\ldots$ \newline
$PP_N \; P2 \; Value$ \newline 

$P1 \; P2 \; =>$ will identify the segment, the following triple define the errors. All the points $PP_i$
is on the same line \newline


How about the tapered unit?

For tapered unit, compute the errors on each sliced image, do the same but with different 
vertices on the faces. \newline \newline

2. How to measure the errors. \newline

For each sliced image, compute a corresponding modeled sliced image. Superposition the 
modeled on on the sliced one. 

For implementation, we might be able to make use of HD computation.

If a line segment is too short ( $Thres_1$ < 2-3 pixels? ), let's ignore it. Namely, do
nothing on this line segment. (We assume this line segment is close enought to the real data)

for each line segment $l$ in model sliced image $I_m$, and the user-defined face wide $w$,
if $l > w$, we have to split the $l$ into pieces of $l_0, l_1, ..., l_N$, such that 
$l_i < w$. Now, let's define an error computing window size, say $S = d$ and do dilation 
for each $l_i$ with the iteration depth of $d$. 
For each dilation, we should mark any data points touched by dilation and remove it. So that
later on, other $l_j$ will not treat this point as its error data.

Before dilation, we should remove those points $p_m$ in $l_i$ which reside on the data points.
During the dilation, if for a point $p_m$ in $l_i$ touched a data point $p$, do the following: 
First, we should remove this point $p_m$ from dilation. If the data point $p$ is not marked
yet, mark it and add the error distance to the line. If $p$ is marked, do nothing.

Problem - dilation is all directions? so may be multiple error computation? (make sense, since
there are some data/noise near this line).

Once the dilation is done, we will obtained accumulated error value $E_a$ for each $l_i$, 
normalized this with $E_a/(L_{l_i} * 2d)$. \newline \newline \newline


??? what if a thick error part like ledger part ???

*. Get a global scale $I_S$ for all models with different resolution.

*. Manually choose a scale? x

*. Get the biggest error for the whole model as nomalized reference.

*. Get the biggest error for a layer as nomalized reference.


%%%%%%%%%%%%%%%%%%%%%%%%%%%%%%%%
\end{document}
