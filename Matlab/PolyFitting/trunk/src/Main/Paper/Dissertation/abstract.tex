Laser range scanners are widely used to acquire accurate scene measurements.
The massive point clouds they generate, however, present challenges to
efficient modeling and visualization.
State-of-the-art techniques for generating 3D models from voluminous
range data is well-known to demand large computational and storage requirements.
In this thesis, attention is directed to the modeling of urban buildings
directly from range data.
We present an efficient modeling algorithm that exploits \emph{a priori}
knowledge that buildings can be modeled from cross-sectional contours
using extrusion and tapering operations.
Inspired by this simple workflow, we identify key cross-sectional slices among
the point cloud.
These slices capture changes across the building facade along the principal axes.
Standard image processing algorithms are used to remove noise, fill holes,
and vectorize the projected points into planar contours.
Applying extrusion and tapering operations to these contours
permits us to achieve dramatic geometry compression, making the resulting
models suitable for web-based applications such as Google Earth
or Microsoft Virtual Earth.
This work has applications in architecture, urban design, virtual city
touring, and online gaming.
We present experimental results on the exterior and interior of urban building
datasets to validate the proposed algorithm.

