\newappendix{Hough Space Plane Detection}{chapt_appen3}{Hough Space Plane Detection}
\label{apdx:HSPD}

The Hough transform was introduced to detect complex patterns 
of points in binary images and became quickly a popular algorithm 
to detect lines and simple curves.
The key idea is to map a difficult pattern detection problem into a simple
peak detection problem in the space of the parameters of the curve.
The Hough transform has several attractive characteristics \cite{book_tv}.
First, it is relatively robust to noise because 
those outlier data points are unlikely to form consistently any geometry shapes, 
and therefore vote to the same bins for a particular parameter.
Second, it is able to compute multiple instances of a dataset in a single pass,
which outperforms most of the pattern matching algorithms based on iteratively
refinement process.

The classical Hough transform \cite{Hough} for line detection in images is based on 
the slope-intercept formulation of a line, i.e., $y = mx + c$, where $(x, y)$ 
is a point on the line, $m$ is the slope and $c$ is the $y-$intercept. 
The Hough transform proceeds by discretizing $m$ and $c$ and for each image point
$p = (x_p, y_p)$, increasing all cells $i$ and $j$ which satisfy $c_i = y_p - m_jx_p$.
The largest accumulator value in the Hough space gives the hypotheses for the line.
However, there is a major weakness with this parameterization: 
because the valid range of $m$ is from $-\infty$ to $+\infty$, 
which cannot be properly discretized, 
the detection gets much poorer as lines become vertical. 

This problem for line detection can be solved by parameterizing the line 
using polar coordinate representation, that is
$xcos\theta + ysin\theta = r$. 
Where $\theta$ is the angle of the normal direction with the $x-$axis 
and $r$ is the perpendicular distance from the origin to the line.
Using this formulation, the Hough space is still 2D but it does not have the discretization
problem of the classical method described above since $0 \le \theta < \pi$.
Consequently, this formulation enables the unbiased detection of vertical lines.

The extension of the principle of classical Hough transform from 2D to 3D for 
plane detection is quite straight-forward. 
A plane is represented by its explicit equation $z = ax + by + c$,
which indicates a 3D Hough space corresponding to $a, b$ and $c$.
This extension suffers from the same problems as its 2D counter part, that is, 
the near-vertical plane detection is hard to be detected reliably due to 
unbounded range of parameters.

The solution for unbiased planar detection in 3D is quite similar to the one for 2D.
The plane is parameterized by its normal direction $\hat{\bf n} = (n_x, n_y, n_z)$ and
its perpendicular distance from the origin $\rho$. 
As there is a constraint on the magnitude of the normal of the plane,
i.e., $||\hat{\bf n}|| = 1$, there are only three degrees of freedom left. 
Because an unconstrained representation with the minimum number of 
parameters is more efficient for the Hough transform, 
one can use spherical coordinates of a unit sphere ($\theta, \phi$) 
for representing the unit normal $\hat{\bf n}$.
\begin{equation}
\hat{\bf n} = (cos\theta sin\phi  \quad sin\theta sin\phi \quad cos\phi) 
\qquad 0 \le \theta < 2\pi \quad 
0 \le \phi \le \pi
\end{equation}
Hence, there are three parameters for 3D Hough space, which are $\theta, \phi$ and $\rho$. 
Each given point ($x_p, y_p, z_p$) in the input point cloud votes for all bins 
$\theta_i, \phi_j $ and $\rho_k$ which satisfy
\begin{equation}
\rho_k = x_pcos\theta_i sin\phi_j \; + \; y_psin\theta_isin\phi_j \; + \; z_pcos\phi_j
\label{eq:ht_plane}
\end{equation}

