\newappendix{Mathematical Morphology}{chapt_appen2}{Mathematical Morphology}
\label{apdx:mm}

In this appendix, the basic morphological operations on binary image 
used in the project are described.
The morphological operations are conducted on set, including 
the binary image $A$, and the structuring element $B$.
$A$ and $B$ are defined on a 2D Cartesian grid, where the 1's are the
elements of those sets. 
We denote by $B_{xy}$ the structuring element after it has been translated
so that its origin is located at the point $(x, y)$. 
The output of a morphological operation is another set.
%%The basic morphological operations are dilation and erosion, 
%%as shown in \Fig{xxx}.

\begin{enumerate}

\item {\bf Dilation}

The dilation operation of $A$ by the structuring element $B$ is defined by:

$ D = A \oplus B = \{(x, y)|A \cap B_{xy} \neq \varnothing\}$

Namely, the output binary image $D$ of the dilating $A$ by structuring 
element $B$ is the set of points $(x, y)$ such that if $B$ is translated
to the origin $(x, y)$, its intersection with $A$ is not empty. 

Essentially, simple dilation (3x3 kernel of $B$) is the process of 
incorporating into the object all the background points that touch it,
leaving it larger in area by one pixel all around its perimeter.
If the object is circular, its diameter increases by two pixels with each dilation.
If two objects are separated by less than 3 pixels at any point, 
they will become connected at that point. 
Therefore, dilation is useful for filling holes in segmented objects.

\item {\bf Erosion}

The erosion operation of $A$ by the structuring element $B$ is defined by:

$ E = A \ominus B = \{(x, y)|B_{xy} \subseteq  A\}$

Namely, the output binary image $E$ of the eroding $A$ by structuring
element $B$ is the set of points $(x, y)$ such that if $B$ is translated
to the origin $(x, y)$, it is completely contained within $A$.

Essentially, simple erosion (3x3 kernel of $B$) is the process of 
eliminating all the boundary points from an object,
leaving it smaller in area by one pixel all around its perimeter.
If the object is circular, its diameter decreases by two pixels with each erosion.
If it narrows to less than 3 pixels thick at any point, 
it will become disconnected at that point. 
Objects with no more than two pixels thick in any direction are eliminated.
Therefore, erosion is useful for removing from a segmented image objects
that are too small to be of interest, such as those noise or outlier data.

\end{enumerate}
