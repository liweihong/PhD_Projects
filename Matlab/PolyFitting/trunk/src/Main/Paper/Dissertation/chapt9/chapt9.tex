\newchapt{Future Work}{chapt9}{Future Work}


%%%%%%%%%%%%%%%%%%%%%%%%%%%%%%%%
%%%%%% Future Work           %%%
%%%%%%%%%%%%%%%%%%%%%%%%%%%%%%%%
\section{Conclusion and Future Work}

This paper has presented an efficient algorithm for lightweight 3D modeling
of urban buildings from range data.
Our work is based on the observation that buildings can be viewed as the
combination of two basic components: extrusion and tapering.
The range data is partitioned into volumetric slabs whose 3D points are
projected onto a series of uniform cross-sectional planes.
The points in those planes are vectorized using an adaptive BPA algorithm
to form a set of polygonal contour slices.
Prominent keyslices are extracted from this set.
Applying extrusion to these keyslices forms lightweight 3D models.
Experimental results on both exterior and interior urban building datasets
have been presented to validate the proposed approach.

We achieve further geometry compression by detecting a series of
slices that coincide with a taper operation.
In the current work, we demonstrate how to infer the taper-to-line
geometry structure.
In future work we will extend this to include the taper-to-point geometry
structure, which appears frequently in Gothic architecture (e.g., churches).
A nice characteristic of this structure is that the image slices will converge
to a point, which is a good inference cue.

Additional future work is to investigate the modeling of the ``follow-me''
geometry structure.
This is a more complicated geometry structure featured in Google SketchUp
that exists when the model can be reconstructed by moving a cross-sectional
unit along a curve trajectory.
We will track the slices to obtain the curve trajectory.
Finally, we will optimize the performance of the BPA vectorization module,
which consumes the bulk of the computation time.

