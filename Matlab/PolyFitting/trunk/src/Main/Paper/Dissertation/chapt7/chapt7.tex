\newchapt{Algorithm Analysis}{chapt7}{Algorithm Analysis}

%%%%%%%%%%%%%%%%%%%%%%%%%%%%%%%%
%%%%%%  Algorithm Analsysis  %%%
%%%%%%%%%%%%%%%%%%%%%%%%%%%%%%%%
\section{Algorithm Analysis}
\label{sec:PE_AA}

\begin{algorithm}
\caption{The Adaptive Hough Transform Algorithm}
\label{alg.AHT}
\begin{algorithmic}[1]
\Procedure {AHTA}{$I$, $\boldsymbol{L}$}
\While {$true$}
\State $\bigcup{(\rho, \theta)} \leftarrow HT (I)$  \Comment{regular hough transform to get a set of lines}
\State $n \leftarrow 0 $
\For {each $(\rho, \theta) \in \bigcup{(\rho, \theta)} $}
     \State $c \leftarrow num(I, \rho, \theta)$  \Comment{compute \# of points falling onto line ($\rho, \theta$)}
     \If {$n < c$}
         \State $n \leftarrow c$
         \State $\rho_{max} \leftarrow \rho$
         \State $\theta_{max} \leftarrow \theta$
     \EndIf
\EndFor
\If {$n > \epsilon $}   \Comment{ sanity check for the line ($\rho_{max}, \theta_{max}$) }
\State $\boldsymbol{L} \leftarrow (\rho_{max}, \theta_{max})$
\State $update(I, \rho_{max}, \theta_{max})$ \Comment{ remove points falling onto line ($\rho_{max}, \theta_{max}$)}
\Else
\State $break$
\EndIf
\If {$\omega(I) < \epsilon$} \Comment{ check how many points left }
\State $break$
\EndIf
\EndWhile
\EndProcedure
\end{algorithmic}
\end{algorithm}

The Adaptive Hough Transform (AHT) algorithm stems from the original Hough transform algorithm.
The function $HT()$ is just a regular Hough Transform (HT), which is $O(L)$ for line detection. 
Here $L$ is the data point in 2D image.  The computation of the number of points covered by 
each line  ($\rho, \theta$) detected by $HT$ is of $O(L)$ complexity.
The sanity check and image update function $update()$ is most of query and comparison operation, which is
$O(1)$ complexity. Because we are dealing with bounded number of data points, and the data points
on the longest line of each iteration are removed for each iteration, the AHT will quickly converge.
Because the 3D point cloud data has been projected to 2D image, the worst case for 3D data points is
$O(N)$, where $N$ is the total number of 3D points.

The space complexity of AHT algorithm is bounded by the space complexity of the $HT()$ algorithm. 
Because we are only working on line detection of 2D images, the space complexity is $O(N_\rho * N_\theta)$, 
where $N_\rho$ and $N_\theta$ represents the number of bins for quantizing the value of $\rho$
and $\theta$ respectively. For example, for a image of size 300x400, if we quantize the length $\rho$
with one bin for each pixel, the $N_\rho$ would be 500. Also, if we quantize the the angle with one
bin for a degree, the $N_\theta$ would be 180. Based on this, the space complexity for this example 
can be easily obtained. The precise of the Hough transform algorithm is depend on the quantization 
of the parameters $\rho$ and $\theta$. The more bins are used for quantization, the more precise the
results will be, which implies more space are needed for the algorithm.


\begin{algorithm}
\caption{The 2D Adaptive Ball-Pivot Algorithm}
\label{alg.ABPA}
\begin{algorithmic}[1]
\Procedure {ABPA}{$I$, $\boldsymbol{L}$}
\State $L \leftarrow \emptyset$
\State $P, P_0 \leftarrow S(I) $ \Comment{compute the $seed$ point assigned to $P$ and $P_0$}
\State $r \leftarrow W$ \Comment{initialize radius $r$ with a large value $W$}
\While {$true$}  \Comment{inital BPA stage}
   \State $append(L, P)$ 
   \State $P_i \leftarrow BPA(P, I, r)$ \Comment{regular 2D BPA}
   \If { $P_i = P_0 $ } 
      \State $break$ \Comment { complete the initial BPA }
   \EndIf
   \State $P \leftarrow P_i$ \Comment { find a new vertex $P$ for the contour $L$}
\EndWhile

\State $r \leftarrow W'$ \Comment{a smaller radius $W'$ for refinement}
\While {$r > \epsilon$} \Comment {iterative BPA refinement}
   \For { each line $\overline{P_iP_j} \in \boldsymbol{L} $}
      \State $L' \leftarrow \emptyset$
      \State $P \leftarrow P_i$
      \While {$true$}
         \State $append(L', P)$ \Comment{construct a sub contour $L'$}
         \State $P_k \leftarrow BPA(P, I, r)$ 
	 \If { $P_k = P_j $ } \Comment{stop when reaching the other point}
	    \State $substitute(L, P_i, P_j, L')$ \Comment{refine $\overline{P_iP_j}$ with $L'$}
	    \State $break$
	 \ElsIf { $isGap(L', P_k)$ }  \Comment{stop when reaching a gap}
	    \State $break$
	 \EndIf
	 \State $P \leftarrow P_k$ \Comment { find a new vertex $P$ for $L'$}
      \EndWhile
   \EndFor
   \State $r \leftarrow r/2$  \Comment{reduce the radius for next iteration}
\EndWhile
\EndProcedure
\end{algorithmic}
\end{algorithm}

The 2D adaptive ball-pivot algorithm (ABPA) is summarized in Algorithm \ref{alg.ABPA}. 
The seed computation $S()$ is to randomly pick up a good starting point, which
is only $O(1)$ complexity. The $append()$ function is to concatenate the control 
points, which is only $O(1)$ complexity. The regular 2D ball-pivot algorithm, $BPA()$,
carries out the geometry computation on each control point based on the size of the
radius of the ball or circle. Basically, for each pivoting of the ball, the area 
in the image covered by this pivoting is checked. If there is a data point contained
by this area, the pivoting ends for this control point. 
The complexity for computing the whole potential pivoting area is $O(n)$???, the
complexity to pick up the mininue angle of the pivoting is $O(1)$, therefore, the 
complexity for the whole 2D $BPA()$ is $O(n)$. 

For the refinement process, each line $\overline{P_iP_j}$ of the boundary generated
in the previous iteration is checked using the same algorithm in $BPA()$. This only
takes $O(n)$ complexity. The $substitute()$ and $isGap()$ function is just to insert
a extra point or a simple algebra computation, the complexity for both operations is 
$O(1)$, Because the round of the refinement process is bounded by a predefined constant 
number, say $c$, the complexity of this refinement is also bounded by $O(c*n) = O(n)$. 

The implementation of the ABPA uses O(m) memory space, where m is the size of the 2D image.
This complexity includes the enque/dequue the control points of the boundary, the 
pivoting geomerty computation and the $substittue()$ and $isGap()$ computations. Since
the user can control the size of slices, memory requirements can be tailored to the
available hardware. 

\begin{algorithm}
\caption{The Keyslice Detection Algorithm}
\label{alg.KSD}
\begin{algorithmic}[1]
\Procedure {Hausdorff}{$I$, $N$}
  \State $K \leftarrow \emptyset$  \Comment{vector to store the indices of keyslices}
  \State $I_r \leftarrow I_0$
  \For { $i \leftarrow 1, N-1 $}
    \State $d_1 \leftarrow dis(I_r, I_i) $
    \State $d_2 \leftarrow dis(I_i, I_r) $
    \State $d_{HD} \leftarrow avg(d_1, d_2) $ \Comment {average Hausdorff distance}
    \If { $d_{HD} > \epsilon $}
       \State $I_r \leftarrow I_i$  \Comment{update the reference image}
       \State $ append(K, i) $
    \EndIf
  \EndFor
  \State \textbf{return} $K$
\EndProcedure

\Procedure {Curvature}{$I$, $N$}
  \State $K \leftarrow \emptyset$  \Comment{vector to store the indices of keyslices}
  \State $V \leftarrow 0$ \Comment {initialize the counter to 0}
  \For { $i \leftarrow 0, N-1 $}
     \State $\bigcup{C} \leftarrow curv(I_i)$ \Comment{curvature computation on image $I_i$}
     \State $\bigcup{c} \leftarrow mapping(\bigcup{C}) $ \Comment{mapping the curvature to indices}
     \For {each $c \in \bigcup{c}$ }
        \State $V(c) \leftarrow V(c) + 1$ \Comment{increase the number of curvature observed}
     \EndFor
  \EndFor
  \For { each $c \in \bigcup{c}$ }
     \If { $V(c) > \epsilon $}  \Comment {validate the curvatures}
	\State $append(K, c)$
     \EndIf
  \EndFor
  \State \textbf{return} $K$
\EndProcedure

\Procedure {KSDA}{$I$, $S$}
\State $K_1 \leftarrow Hausdorff(I, S)$ \Comment{keyslices computed by Hausdorff distance}
\State $K_2 \leftarrow Curvature(I, S)$ \Comment{keyslices computed by Curvature}
\State $K \leftarrow merge(K_1, K_2)$
\EndProcedure
\end{algorithmic}
\end{algorithm}

The idea of keyslice detection is described in Algorithm \ref{alg.KSD}. 
Essentailly, the algorithm consists of two independent procedures. 
The function $Hausdorff()$ is conducting Hausdorff distance based keyslice
detection. A lookup table $t$ is created for the implementation of $Hausdorff()$. 
This lookup table is created by computing for each pixel $p$ in the reference image $I_r$,
how far is $p$ to the nearest data point. If $p$ iteself is a data point, the distance is 0.
The complexity of constructing $t$ bounded by $O(w*h)$, where $w$ and $h$ are the image
width and height respectively. Once $t$ is computed, the function $dis()$ is only
$O(1)$ complexity which basically doing queuing from $t$. Therefore, the function $Hausdorff()$
takes $O(n)$ complexity. 
For procedure $Curvature()$, the function $curv()$ which computes curvatures
in the orthogonal direction is of $O(n)$ complexity. The $mapping()$ funciton and
the remaining steps mainly consists of couple of algebra computation, which takes $O(1)$ complexity. 
Therefore the procedure $Curvature()$ is of $O(n)$. Overall, the algorithm $KSDA()$ takes
$O(n)$ computation complexity for keyslice detection. 

For the space complexity, the keyslice detection algorithm is similar to adaptive ball-pivot
algorithm and uses $O(m)$ memory for computation. Again, the memory usage can be tailored to 
available hardware based on user settings.


