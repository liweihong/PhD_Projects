\ifx\twocol\undefined
\documentclass[12pt,letterpaper]{article}
\newcommand{\twocol}{0}
\else
\documentclass[8pt,twocolumn,letterpaper]{article}  % TWO COLUMN VERSION
\renewcommand{\twocol}{true}
\fi

\usepackage{amsmath}
\usepackage[pdftex]{color,graphicx}
\usepackage{verbatim}
\usepackage{setspace}
\usepackage{subfig}
\usepackage{algorithm}
\usepackage{algorithmicx}
\usepackage{algpseudocode}
\usepackage{ifthen}

%%%%%%%%%%%%%%%%%%%%%%%%%%%%%%%%
%%%%%%   GLOBAL SETTING  %%%%%%%
%%%%%%%%%%%%%%%%%%%%%%%%%%%%%%%%
\onehalfspace

\begin{document}
%%%%%%%%%%%%%%%%%%%%%%%%%%%%%%%%
%%%%%%   ABSTRACT      %%%%%%%%%
%%%%%%%%%%%%%%%%%%%%%%%%%%%%%%%%
\section{Abstract}
The state-of-the-art techniques for 3D modeling of range data are usually heavy-duty and
suffer from large-scale datasets.
This paper presents an efficient algorithm to reconstruct 3D models of urban buildings by
making use of a \emph{prior} knowledge that buildings can be created through two basic
operations, \emph{extrude} and \emph{taper}, on some fundamental geometry shapes.
The range data is first projected into a series of 2D images from which
key images representing the salient features are selected.
The basic geometry shapes are then obtained from these raster key images via vectorization.
Depends on the 3D characteristics, the final model is reconstructed by either
extrude or taper operation on these basic shapes.
The proposed algorithm is of multi-resolution, and even for a low resolution model,
the sharpness of the raw data is preserved, which
outperforms most of the existed approximation algorithms.
The contribution of this work is that it combines the benefits of a \emph{prior} knowledge on
urban buildings and the light-weight 2D image processing techniques for 3D modeling and therefore
leads to an economical geometry compression approach on range data.
This work can boost web-based 3D applications, including virtual city touring, online gaming, etc.
The experimental results on both exterior and interior urban building datasets are presented to
validate the proposed approach.

\ifthenelse{\equal{\twocol}{true}}{
%\newpage % TWO COLUMN VERSION
}{
\newpage % TWO COLUMN VERSION
}
\section{Practical Aspects and Timeline}

In the processing of implementing the algorithm to obtain the preliminary results, we gained in-depth
knowledge of the 3D tools and related programming language. The whole project consists of over 20000
lines of C++ code and over 1000 lines of Ruby code of Google SketchUp plug-ins for model visualization.
Based on the work has been done and this code base, we believe that the remaining research will
progress fast and the dissertation defense will be in July 2009.

The remaining research problems and the estimated timeline are listed in the following:
\begin{enumerate}

\item Investigate modeling of the tapered to a point geometry structure

We have successfully infered the geometry structure of tapered to a line. As a further step, we will study
how to infer the geometry structure of tapered to a point which appears frequently in Gothic architecture,
like churches.
Our plan is to first segment out the region dedicated to this special structure which will
be then projected into a series of sliced images to be analyzed.
This segmentation task can be done based on the connected components in the sliced 2D images.
Unlike the tapered to a line, we are not able to infer this structure by inferring extruded
structure from the orthogonal directions.
However, a nice characteristic of this structure is that
the sliced image will converge to a point, which is good clue for inference.
Without loss of generality, we assume that the sliced images of this special structure
can be vectorized by a closed polygon.
Let $S_i$ and $S_j$ be two consecutive sliced images with $i < j$, that is, $S_i$
has a smaller boundary than $S_j$. For this special structure, $S_i$ will be growing up
to cover $S_j$ by iterative dilation operations. By doing this, we can quickly locate the
bottom and top (a small region representing a point) slices of this special geometry
structure and therefore reconstruct it based on these two slices.
The difference between this structure and the structure of tapered to a line is whether
$S_i$ and $S_j$ have any overlapping or shared parts or not, which is easy to check.

The time needed to do this is about 2 weeks.
One and a half weeks will be dedicated to infer the underline structure.
The remaining time is used to create the model.

\item Investigate modeling of the ``follow-me'' geometry structure

This is a more complicated geometry structure compared to the tapered structure mentioned above.
This type of structure exists when the underline geometry structure can be reconstructed by
moving a basic geometry unit along a curve trajectory.
Therefore we have to carry out a tracking computation to obtain the curve trajectory.
To do this, we will first capture the base geometry unit from a projected 2D reference slice image $S_i$.
This base geometry is a very important initial stage because it is used as a template $A$
for the further tracking process.
Alternatively, to make this more precise or reliable, we could also get the base geometry part
of a reference image from user input.
Once the template $A$ is defined, we can start the tracking from the reference image along
two oppose directions, that is $i, i+1, \ldots$ and $i, i-1, \ldots$.
Because of the dense projection, the template matching tracking process only need to search
in a small local region.
This tracking process ends when no pattern was found in the new sliced images, which means the
boundary of this special structure is reached.

The estimated time needed to do is about 2 weeks.
One and a half weeks will be dedicated to the tracking of the base geometry
and the remaining time is used to compute the final model from the tracking results.

\item Performance evaluation with other techniques

We will conduct more advanced performance evaluation on the proposed approach, especially the
comparison on state-of-the-art techniques.
First of all, we will carry out the comparison on the high resolution models
generated with our proposed method and those generated by using ball-pivot algorithm.
The BPA is a very efficient algorithm and can generate precise models from range data with
a small ball radius.
To do the comparison, the number of faces or triangles in these high resolution models
from both methods will be approximately the same level.
The metric for comparison will include both objective and subjective criteria. The objective
comparison will measure the errors between the models and the raw range data.
And a subjective measurement will rely on sharpness of the models for the visualization.
On the other hand, we will also carry out the comparison on the low resolution models
generated by our methods and others, such as qslim approach.
The same comparison metrics used for high resolution models will be used here.

The estimated time for this is 2 weeks.
One week is needed to download and run others tools, such as qslim.
And another week is used to conduct the comparisons.

\item Carry out experiments on more complicated datasets

Currently we have only applied the proposed approach on one set of range data, which is a
exterior scanning of the buildings.
In order to test the effectivity and reliability the proposed algorithm,
we will carry out the experiments on more complicated point cloud datasets, such as
interior scanning of a large hall, where more sophisticated structures exist.
Roughly we will try the proposed approach on about half dozen of large-scale datasets to
validate our techniques.

The estimated time for this work is about 3 weeks.
The dataset may not be the same size but on average two datasets will be processed in each week.


\end{enumerate}


%%%%%%%%%%%%%%%%%%%%%%%%%%%%%%%%
\end{document}
